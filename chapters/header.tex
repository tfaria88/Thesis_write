%
% Master thesis
% ===========================================================================
%

%
% A. DOCUMENTCLASS
% ---------------------------------------------------------------------------
%

%
%  1. Define the document class.
%     We use the KOMA-Script class 'scrbook' for a book.
%     Define the Layouts
%
\documentclass[%
    pdftex,%              We use PDFTex because we will create a PDF.
    a4paper,%             We use A4 paper.
    oneside,%             Single-sided printing.
    11pt,%                Big font, more suitable for A4.
    %halfparskip,%        Half line spacing between paragraphs.
    headsepline,%         Line after the header.
    bibtotocnumbered%     Numbered insert the bibliography in the table of contents.
    %idxtotoc%            Adding an index into the table of contents.
]{scrbook}

%
%  2. Set the character encoding of the document and the character set.
%     We use 'latin1' (ISO-8859-1) for the document,
%     and the 'T1' coding for the font.
%
%\usepackage[latin1]{inputenc} %for unixoide and also Windows systems
\usepackage[ansinew]{inputenc} %for windows
\usepackage[T1]{fontenc}

%
%  3. Load the package for the index creation.
%
\usepackage{makeidx}

%
%  4. Load the package for localization in German.
%     With 'ngerman' we use new German spelling.
%
\usepackage[english]{babel}


%
%  5. Load package for quotation marks.
%     We set the style to 'swiss', and so use the Swiss quotation marks.
%
\usepackage[style=swiss]{csquotes}


%
%  6. Package for advanced table properties.
%
\usepackage{array}

%
%  7. Package to be able to embed graphics in the document.
%     Im the PDF is possible to embed GIF, PNG, und PDF graphics.
%
\usepackage[pdftex]{graphicx}

%
%  8. Packages for mathematical typesetting.
%
\usepackage{amsmath}
\usepackage{amssymb}
\usepackage{dsfont}
\usepackage{mathtools}
\usepackage{mathrsfs}
\usepackage{txfonts}

%
%  9. Package to rotate parts of the text.
%
\usepackage{rotating}

%
% 10. Package for colors at different locations.
%     We define additional named colors.
%
\usepackage{color}

%
% 11. Package for special PDF features.
%
\usepackage[%
    pdftitle={xxx},%                         Title of the PDF document.
    pdfauthor={Name des Autors},%            Autor of the PDF document.
    pdfsubject={Inhalt},%                    Subject of the PDF document.
    pdfcreator={Texmaker, LaTeX with hyperref and KOMA-Script},% Producers of the PDF document.
    pdfkeywords={Keywords,xxx,sdadf,},%      Keywords for the PDF. (For searching algorithms)
    pdfpagemode=UseOutlines,%                Show table of contents when opening
    pdfdisplaydoctitle=true,%                Show document title instead of file name.
    pdflang=en%                              Language of the document.
]{hyperref}

%
% 12. Using the option 'savemem' we delay the loading of the
%     different programming languages ??at a later date.
%
\usepackage[savemem]{listings}


% 13. Use of the font 'Latin Modern Family'.
%     Use this package if you compile DML itself.
%

\usepackage{lmodern}

%
% 14. Loading of Typewriter font LuxiMono.
%
%\usepackage[scaled=.85]{luximono}

%
% 15. Layout
%
% The header and footer layout fit better for the document class KOMA-Script (scrbook) as the Pake fancyhdr, otherwise fairly equivalent
\usepackage{scrpage2}   
\usepackage[a4paper,vmargin={27mm,23mm},hmargin={25mm,25mm}]{geometry}

%
% 15. Compact list
%
\usepackage{paralist}         


% 
% B. SETTINGS
% ---------------------------------------------------------------------------
%

%
%  1. Define your own named colors.
%     For the use later in the document, we define individual named colors.
%
\definecolor{LinkColor}{rgb}{0,0,0.5}
\definecolor{ListingBackground}{rgb}{0.85,0.85,0.85}

%
%  2. KOMA-Script Option, Line break in captions.
%
\setcapindent{1em}

%
%  3. The style of the header and footer.
%     We activate with 'headings' ongoing title page.
%
%\pagestyle{headings}

%
%  4. The style of the headings on normal font.
%     We use for the headers the same font as the text.
%
\setkomafont{sectioning}{\normalfont\bfseries}       % Title with normal font
\setkomafont{captionlabel}{\normalfont\bfseries}     % Bold labels 
\setkomafont{pageheadfoot}{\normalfont\itshape}      % Italic page title
\setkomafont{descriptionlabel}{\normalfont\bfseries} % Bold description title

%
%  5. Color settings for the links in the PDF document.
%
\hypersetup{%
    colorlinks=true,%        Enabling colored links in the document (no frames)
    linkcolor=LinkColor,%    Specify the color.
    citecolor=black,%        Specify the color.
    filecolor=LinkColor,%    Specify the color.
    menucolor=LinkColor,%    Specify the color.
    urlcolor=LinkColor,%     Color from URL's in the document.
    linkcolor=black,%        Color of internal links
    bookmarksnumbered=true%  Show heading numbering in PDF contents.
}

%
%  6. Settings for the 'listings' package.
%
\lstloadlanguages{TeX} %     TeX language loading, it is required for option 'savemem'
\lstset{%
    language=[LaTeX]TeX,     % The language of the source code is TeX
    numbers=left,            % Row numbers on the left
    stepnumber=1,            % Numbering of each line.
    numbersep=5pt,           % 5pt distance for the source code
    numberstyle=\tiny,       % Character size 'tiny' for the numbers.
    breaklines=true,         % Wrap lines if necessary.
    breakautoindent=true,    % After the line break row indent..
    postbreak=\space,        % Break up at spaces.
    tabsize=2,               % Tab size 2.
    basicstyle=\ttfamily\footnotesize, % Non-proportional font, small for the source code.
    showspaces=false,        % Does not show spaces.
    showstringspaces=false,  % Do not show spaces also in strings ('').
    extendedchars=true,      % Show all characters from the Latin1 font.
    backgroundcolor=\color{ListingBackground}} % Background color of the source code.
    
%
% Flowcharts
%
\usepackage{import}
\usepackage{scrhack}
\usepackage{floatrow}
\floatsetup[figure]{font=normalsize}   %{font=small} 
\usepackage{subfigure}                      % Several pictures next to each.
%
% Literature and other references
%    

% \usepackage{cite}              % Sorted and summarized citation numbers 
% \usepackage{varioref}          % Improved references
%\usepackage{natbib}             % You need for natdin.bst (references in the text by name and year)

%\setcitestyle{square,aysep={}}  %Square brackets to quote

\usepackage{longtable}             % Provides tables that exceed the page break (see List of Symbols)
%

%
% Your own settings
%    
\usepackage{indentfirst}         % Indent first line after heading
\parindent 0pt
\parskip 3pt

\usepackage{pdfpages}            % Inserting pdf pages

\usepackage{exscale}             % Customizing sums and integral characters to the Font Size:

% C. NEW MACROS AND SURROUNDINGS
% ---------------------------------------------------------------------------


%
%  1. Setting for change list with a special bullet character.
%
\newenvironment{ListChanges}%
    {\begin{list}{$\diamondsuit$}{}}%
    {\end{list}}

%
%  2. Replacement for the \ LaTeX and \ TeX commands for correct representation.
%     We use the 'Latin Modern Family' ('lm') as font, as this compared to 
%     'Computer Modern' ('cm') also offer PostScript files, which leads to a more beautiful
%     presentation in PDF.

\newcommand{\DMLLaTeX}{{\fontfamily{lmr}\selectfont\LaTeX}}
\newcommand{\DMLTeX}{{\fontfamily{lmr}\selectfont\TeX}}

\def\AmS{$\mathcal{A}$\kern-.1667em\lower.5ex\hbox{$\mathcal{M}$}\kern-.125em$\mathcal{S}$}
\def\AmSmath{\AmS{}math}

%----------------------------------------------------------------------------
%%%%%%%%%%%%%%%%%%%%%%%%%%%%%%%%%%%
%
% New- and Renew-Commands
%
%%%%%%%%%%%%%%%%%%%%%%%%%%%%%%%%%%%


%\renewcommand{\headfont}{\normalfont\sffamily}             % Kolumnentitel serifenlos
%\renewcommand{\pnumfont}{\normalfont\sffamily\bfseries}    % Page numbers typewriter and in bold 
\pagestyle{scrheadings}                                                                         
\clearscrheadings                                           % To delete page number below
\clearscrplain
\ihead[]{\headmark}                                         % Columns title always on top inside
\ohead[\pagemark]{\pagemark}                                % Page numbers always on top outside
\lefoot[]{} 
\rofoot[]{}                                                 % Delete page numbers in the footer

\renewcommand{\bibname}{Literatur}                          % References to literature 
\renewcommand{\figurename}{Bild}                            % Figure to image
\renewcommand{\listfigurename}{Bildverzeichnis}

\newcommand {\jkarray}[1]{\ensuremath{\underline{#1}}}     % so is a unique command
\newcommand {\jkmatrix}[1]{\ensuremath{\underline{\underline{#1}}}} % so is a unique command
\newcommand {\einheit}[1]{\ensuremath{\mathrm{\left[#1\right]}}} 

\newcommand {\lived}[2]{($\ast$#1, $\dagger$#2)}  %

% andere Schrift
%\renewcommand{\familydefault}{\sfdefault}
%\usepackage{helvet}


\def\TReg{\textsuperscript{\textregistered}}
\def\TCop{\textsuperscript{\textcopyright}}
\def\TTra{\textsuperscript{\texttrademark}}
%
% D. ACTIONS
% ---------------------------------------------------------------------------
%

%
%  1. Create the index.
%
\makeindex

%
% E. HYPHENATION
% ---------------------------------------------------------------------------
%

\hyphenation{De-zi-mal-trenn-zeichen In-stal-la-ti-ons-as-sis-tent}

%%%%%%%%%%%%%%%%%%%%%%%%%%%%%%%%%%%
%
%Style definition of listings, setting of parameters in different packages
%
%%%%%%%%%%%%%%%%%%%%%%%%%%%%%%%%%%%
\usepackage{listings}              % Code application environment
\usepackage{multicol}
\usepackage[svgnames,table,hyperref]{xcolor}


% svg figures
\newcommand{\executeiffilenewer}[3]{%
\ifnum\pdfstrcmp{\pdffilemoddate{#1}}%
{\pdffilemoddate{#2}}>0%
{\immediate\write18{#3}}\fi%
}
\newcommand{\includesvg}[1]{%
\executeiffilenewer{#1.svg}{#1.pdf}%
{inkscape -z -C --file=#1.svg %
--export-pdf=#1.pdf --export-latex}%
\input{#1.pdf_tex}%
}

\newcommand{\includegp}[1]{%
%\immediate\write18{wgnuplot #1.gp}%
\executeiffilenewer{#1.plt}{#1.eps}%
{gnuplot #1.plt}%
\ifpdf\executeiffilenewer{#1.eps}{#1.pdf}%
{epstopdf #1.eps}\fi%
\input{#1.tex}%
}