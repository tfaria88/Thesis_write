%%%%%%%%%%%%%%%%%%%%%%%%%%%%%%%%%%%%%%%%%%%%%%%%%%%%%%%%
%%%%                                              %%%%%%
%%%%  Author: Name des Autors                     %%%%%%
%%%%                                              %%%%%%
%%%%  Beschreibung:                               %%%%%%
%%%%                                              %%%%%%
%%%%%%%%%%%%%%%%%%%%%%%%%%%%%%%%%%%%%%%%%%%%%%%%%%%%%%%%

\chapter{LAMMPS Code Modifications}
\label{chap:chapter_3}

LAMMPS is a molecular dynamics code that models particles in a liquid, solid or gaseous state\cite{lammps_manual}. It can model atomic and polymeric systems using a variety of force fields and
boundary conditions. Even that code is primarily aimed for molecular dynamics simulations of atomistic systems, it provides a fully parallelized framework for particle simulations
governed by Newton's equations of motion. Due to its particle nature, SPH is totally compatible with the existing code architecture and data structures present in LAMMPS. There is 
an add-on module in LAMMPS that includes the SPH module into the code.\par

\section{LAMMPS SPH module test case}
\label{sec:section_1}

First, it was necessary to perfom a validation case to have a better understanding of the code usage and to ensure the SPH-package works successfully. The case was taken from 
the SPH-USER Documentation from LAMMPS documentation\cite{ganzenmuller_implementation_2011}. This simulation consists on a shear cavity flow, which is a standard test for a laminar
flow profile. It was considered a 2D square lattice of fluid particles with the top edge moving at a constant speed at a constant speed of $10^-3m/s$. The other three edges are kept
stationary. The driven driven fluid inside is represented by Tait's equation of state \cite{neece_tait_1968} with Morris' laminar flow viscosity. and the kinematic viscosity used is
$\nu=10^-6m^2/s$. A steady-state flow is reached after some thousand cycles and it is shown in Figure~\ref{fig:Bild1}(a). A centerline in the cavity was taken to select some particles
to analyse their velocities (Figure~\ref{fig:Bild1}(b)). The velocity profile along the vertical centerline of the cavity 
agrees pretty well qualitatively with a Finite Difference solution and the results achieved in the SPH-USER documentation (Figure~\ref{fig:Bild2}). The input script is in ~\ref{app:NURBSVolumenelement}.


\begin{figure}[ht]
\centering
  \begin{footnotesize}
  \includesvg{images/cavity_simu}
  \caption[(a) Simulation snapshot of the shear driven fluid filled cavity. Particles are colored according to their kinetic energy. (b) Set of particles located in the cavity centerline used to calculate the velocity profile.]{(a) Simulation snapshot of the shear driven fluid filled cavity. Particles are colored according to their kinetic energy. (b) Set of particles located in the cavity centerline used to calculate the velocity profile.}
  \label{fig:Bild1}
  \end{footnotesize}
\end{figure} 



\begin{figure}[H]
\centering
  \begin{footnotesize}
  \includesvg{images/cavity_graphics}
  \caption[(a) Velocity profile along centerline of the cavity with SPH and FDM solutions from \cite{neece_tait_1968} , (b) Simulation results for velocity profile along centerline ]{(a) Velocity profile along centerline of the cavity with SPH and FDM solutions from \cite{neece_tait_1968} , (b) Simulation results for velocity profile along centerline }
  \label{fig:Bild2}
  \end{footnotesize}
\end{figure} 



\section{Create a swimmer in LAMMPS}
\label{sec:section 2}

LAMMPS is ready to create particles and bonds between particles, but there is no specific routine to create swimmers. The desired swimmer structure is described in Figure~\ref{fig:Bild2.7}
and to create this design it is not so straightfoward. A new routine was created as an input file to introduce swimmers in the simulation according to the necessary input parameters
required by the code. This input file was called \textit{"addswimmer"} and wrote in AWK programming language as it is very convenient for easily writting data files. It has 
the capability of adding one or more swimmers in any position inside the simulation box.\par

There are some variables which need to be initially defined in this file to create the swimmer. The first variable is the number of swimmers present in the simulation, and for each swimmer
it must be defined the $x$ and $y$ cordinates of the swimmer starting point, and this point is the first particle of the tail (from left to the right) in the lower corner.
The next parameters to be settled are the tail length and the head length, where the first is a function of the total swimmer length ( $2/9$ of the total length) and the second is
a free parameter, here defined as three particles length. With those initial parameters the initial structure is created as described in Figure~\ref{fig:Bild2.5}, some extra functions
have the aim to remodel the swimmer and to output the necessary data for LAMMPS to use as input parameters. The function \textit{xy2id} transforms the particle $x$ and $y$ coordinates
to the particle ID, as this data is essencial for the output data to create the bonds. The function \textit{is$\_$on$\_$grid} is used to smooth the head format, deleting the corner
particles from the square grid in the head. Function \textit{bond$\_$filter} adds filters to change the bonds configuration in the swimmer head. The next set of functions have the aim
to differ the bond types from the active tail surface (active bonds), passive tail surface and head (passive bonds) and the internal bonds (strong bonds). One special function called
\textit{add$\_$line$\_$to$\_$change$\_$type} differs the bond type of the head flesh region to the rest of the passive bonds. The last function to be used is the \textit{create$\_$swimmer}
which attach all the previous functions and creates the desired swimmer configuration and it outputs the LAMMPS data file containing the number of atoms, number of bonds, number of 
atom types and simulation box size (defined in a initial input file outside \textit{"addswimmer"}), and a list of atoms, velocities and bonds. The code file of \textit{"addswimmer"}
and one example of output fie created by it is available in Appendix ~\ref{app:addswimmer}.

\section{Bond Style Harmonic Shift}
\label{sec:section 3}





 