%%%%%%%%%%%%%%%%%%%%%%%%%%%%%%%%%%%%%%%%%%%%%%%%%%%%%%%%
%%%%                                              %%%%%%
%%%%  Author: Name des Autors                     %%%%%%
%%%%                                              %%%%%%
%%%%  Beschreibung:                               %%%%%%
%%%%                                              %%%%%%
%%%%%%%%%%%%%%%%%%%%%%%%%%%%%%%%%%%%%%%%%%%%%%%%%%%%%%%%

\chapter{LAMMPS Code Modifications}
\label{chap:chapter_3}

\section{Section}
\label{sec:section_1}

LAMMPS is a molecular dynamics code that models particles in a liquid, solid or gaseous state\cite{lammps_manual}. It can model atomic and polymeric systems using a variety of force fields and
boundary conditions. Even that code is primarily aimed for molecular dynamics simulations of atomistic systems, it provides a fully parallelized framework for particle simulations
governed by Newton's equations of motion. Due to its particle nature, SPH is totally compatible with the existing code architecture and data structures present in LAMMPS. There is 
an add-on module in LAMMPS that includes the SPH module into the code.\par
First, it was necessary to perfom an example case to have a better understanding of the code usage and to ensure the SPH-package works successfully. The example case was taken from 
the SPH-USER Documention from LAMMPS documentation\cite{ganzenmuller_implementation_2011},


\begin{figure}[ht]
\centering
  \begin{footnotesize}
  \includesvg{images/cavity}
  \caption[Initial swimmer structure configuration (upper) and modified final swimmer structure (lower)]{Initial swimmer structure configuration (upper) and modified final swimmer structure (lower)}
  \label{fig:Bild3}
  \end{footnotesize}
\end{figure} 





\section{Section}
\label{sec:section 2}


